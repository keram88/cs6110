%%%%%%%%%%%%%%%%%%%%%%%%%%%%%%%%%%%%%%%%%
% Short Sectioned Assignment
% LaTeX Template
% Version 1.0 (5/5/12)
%
% This template has been downloaded from:
% http://www.LaTeXTemplates.com
%
% Original author:
% Frits Wenneker (http://www.howtotex.com)
%
% License:
% CC BY-NC-SA 3.0 (http://creativecommons.org/licenses/by-nc-sa/3.0/)
%
%%%%%%%%%%%%%%%%%%%%%%%%%%%%%%%%%%%%%%%%%

%----------------------------------------------------------------------------------------
%	PACKAGES AND OTHER DOCUMENT CONFIGURATIONS
%----------------------------------------------------------------------------------------

\documentclass[paper=a4, fontsize=11pt]{scrartcl} % A4 paper and 11pt font size

\usepackage[T1]{fontenc} % Use 8-bit encoding that has 256 glyphs
\usepackage{fourier} % Use the Adobe Utopia font for the document - comment this line to return to the LaTeX default
\usepackage[english]{babel} % English language/hyphenation
\usepackage{amsmath,amsfonts,amsthm} % Math packages

\usepackage{sectsty} % Allows customizing section commands
\allsectionsfont{\centering \normalfont\scshape} % Make all sections centered, the default font and small caps

\usepackage{fancyhdr} % Custom headers and footers
\pagestyle{fancyplain} % Makes all pages in the document conform to the custom headers and footers
\fancyhead{} % No page header - if you want one, create it in the same way as the footers below
\fancyfoot[L]{} % Empty left footer
\fancyfoot[C]{} % Empty center footer
%\fancyfoot[R]{\thepage} % Page numbering for right footer
\renewcommand{\headrulewidth}{0pt} % Remove header underlines
\renewcommand{\footrulewidth}{0pt} % Remove footer underlines
\setlength{\headheight}{1pt} % Customize the height of the header

\numberwithin{equation}{section} % Number equations within sections (i.e. 1.1, 1.2, 2.1, 2.2 instead of 1, 2, 3, 4)
\numberwithin{figure}{section} % Number figures within sections (i.e. 1.1, 1.2, 2.1, 2.2 instead of 1, 2, 3, 4)
\numberwithin{table}{section} % Number tables within sections (i.e. 1.1, 1.2, 2.1, 2.2 instead of 1, 2, 3, 4)

\setlength\parindent{0pt} % Removes all indentation from paragraphs - comment this line for an assignment with lots of text

%----------------------------------------------------------------------------------------
%	TITLE SECTION
%----------------------------------------------------------------------------------------

%\newcommand{\horrule}[1]{\rule{\linewidth}{#1}} % Create horizontal rule command with 1 argument of height

\title{	
\normalfont \normalsize 
\textsc{CS 6110} \\ % Your university, school and/or department name(s)
%\horrule{0.5pt} \\[0.4cm] % Thin top horizontal rule
%\huge N Queens \\ % The assignment title
%\horrule{2pt} \\[0.5cm] % Thick bottom horizontal rule
}

\author{Mark S. Baranowski} % Your name

\date{\normalsize\today} % Today's date or a custom date

\begin{document}

\maketitle % Print the title

%----------------------------------------------------------------------------------------
%	PROBLEM 1
%----------------------------------------------------------------------------------------
The derivation is given below. Other than the implicit encoding of one queen per row, and the diagonals only banning queens on the lower rows, the derivation is mechanical. The variable $q_{r,c}$ denotes whether a queen is on the square $(r,c)$.

\begin{align*}
C_{r,c} & \equiv \bigwedge_{i=1, i \ne r}^n \neg q_{i,c} & \text{(Column constraint)}\\
DR_{r,c} & \equiv \bigwedge_{i=1}^{\min(n-r, n-c)}\neg q_{r+i, c+i} & \text{(Right diagonal constraint)}\\
DL_{r,c} & \equiv \bigwedge_{i=1}^{\min(n-r, c)}\neg q_{r+i, c-i} & \text{(Left diagonal constraint)}\\
R_{r} &\equiv \bigvee_{c=1}^n q_{r,c} \wedge C_{r,c}\wedge DR_{r,c} \wedge DL_{r,c} &\text{(Row constraint)}\\
NQ &\equiv \bigwedge_{r=1}^nR_r & \text{(N Queens constraint)}
\end{align*}
At the top level, $NQ$ is already in CNF, however $R_r$ is not in CNF. Introduce new ``phantom`` variables $p_{r,c}$ where
$p_{r,c}\leftrightarrow q_{r,c} \wedge C_{r,c}\wedge DR_{r,c} \wedge DL_{r,c}.$
Rewrite $R_r$ as $R_r \equiv \bigvee_{c=1}^n p_{r,c}.$
Two new constraints are introduced
\begin{align*}
&p_{r,c} \rightarrow q_{r,c} \wedge C_{r,c} \wedge DR_{r,c} \wedge DL_{r,c}\\
\equiv&\neg p_{r,c} \vee (q_{r,c} \wedge C_{r,c} \wedge DR_{r,c} \wedge DL_{r,c})\\
&q_{r,c} \wedge C_{r,c} \wedge DR_{r,c} \wedge DL_{r,c}\rightarrow p_{r,c}\\
\equiv&\neg q_{r,c} \vee \neg C_{r,c} \vee \neg DR_{r,c} \vee \neg DL_{r,c} \vee p_{r,c}.
\end{align*}
From here, apply the distributive rule and the formula is in CNF.

The tool can be built with {\tt make} and then run with {\tt ./nqueens \#}. To get all solutions, run {\tt ./nqueens \# -1}.
\end{document}


































